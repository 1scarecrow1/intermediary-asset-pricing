{
\fontsize{12}{14}\selectfont
\textnormal{Replicating} \textit{Intermediary asset pricing: New evidence from many asset classes}
}

\bigskip

{
\fontsize{10}{12}\selectfont
James, Young Jin Song, Jaehwa Youm, Monica Panigrahy, and Jacob Simeral
}

For our project, we were tasked with replicating 4 tables from the paper Intermediary asset pricing: New evidence from many asset classes. This paper aims to show how shocks to the equity capital ratio of financial intermediaries, specifically Primary Dealer counterparties of the New York Federal Reserve, exhibit substantial explanatory power for cross-sectional variance in expected returns across diverse asset classes, indicating the pivotal role of intermediary capital risk factor in understanding asset pricing dynamics. We split our work up amongst our selves to accomplish this task. Jacob and Young Jin worked on Tables 1, 2, and A,1. Monica and Jaehwa worked on Table 03.
We found some success in replicating some of the more recent numbers but found much difficulty in determining the exact methodology used by the original authors to determine which holding company to use. We had two different attempts at this, the first was to choose the holding company that held the primary dealer for the longest period of time in a specified period - this data is in ticks.csv. The second method was to use the current holding company for a given primary dealer - this data is in ticksv2.csv. The third, which we did not have time to do, would be to have sub-periods for each holding company of a given primary dealer, which would likely produce the most accurate results.
To pull our data, we used the Compustat database through our WRDS subscription. Specifically, we used the quarterly fundamentals table from Compustat.
